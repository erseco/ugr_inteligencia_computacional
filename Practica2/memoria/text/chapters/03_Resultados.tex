\chapter{Resultados}

Una vez desarrolladas las diferentes variantes del algoritmo pasamos a realizar diferentes ejecuciones cambiado los parámetros.


\begin{table}[ht!]
\centering
\label{resultados_algunas_bd}
\begin{tabular}{ l | r r r | r}
 & \textbf{Básico} & \textbf{Baldwiniano} & \textbf{Lamarckiano} & \textbf{Optima}  \\ \hline
  
chr12a  &  11560 & 9552 & 9552 & 9552 \\
nug12   &  586 & 578 & 578 & 578 \\
nug20   &  2656 & 2570 & 2570 & 2570 \\
chr20a  &  3174 & 2244 & 2196 & 2192 \\
bur26a  &  5438534 & 5426670 & 5426670  & 5426670 \\
bur26b  &  3826071 & 3817852 & 3817852  & 3817852 \\
lipa40a  &  32051 & 31538 & 31538 & 31538 \\
tai60a  &  7756658 & 7340324 & 7296884 & 7205962 \\
\textbf{tai256c} &  \textbf{45555622} & \textbf{44980610} & \textbf{44922898}  & \textbf{44759294}\\

\end{tabular}
\caption{Resultados sobre varios conjuntos de datos}
\end{table}

En la tabla \ref{resultados_algunas_bd} podemos ver los resultados obtenidos al ejecutar nuestro algoritmo sobre varios de los conjuntos de datos proporcionados. En todos se han utilizado los mismos parámetros para los tres algoritmos como se puede ver en la siguiente tabla.

\begin{table}[ht!]
\centering

\label{parametros}
\begin{tabular}{ l r }
Tamaño de la población: & 100\\ \hline
Número de generaciones: & 400\\ \hline
Probabilidad de mutación por individuo: & 0.3\\ \hline
Probabilidad de mutación por gen: & 0.002\\ \hline
Tamaño del torneo: & 4\\ \hline

\end{tabular}
\caption{Parámetros utilizados}
\end{table}

Como hemos visto en la tabla anterior (ver \ref{resultados_algunas_bd}), la variante Lamarckiana es la que mejores resultados genera, seguido de cerca por la variante Baldwiniana que también se acerca al óptimo en la mayoría de conjuntos con los que hemos probado. Podemos empezar a apreciar diferencias en conjuntos grandes como el \textbf{tai256c} donde de nuevo los resultados son muy similares entre las dos variantes, quedando siempre por encima de algoritmo genético básico. Se adjunta el volcado del programa en la ejecución final con los tres algoritmos.


\begin{table}[ht!]
\centering

\label{resultados:e1}
\begin{tabular}{ l }
Type: LAMARCKIAN\\
Solution: [109, 77, 30, 86, 83, 120, 111, 246, 68, 65, 232, 25, 19, 34, 55, 137, 181, 116, 47, 48, 5, 72, 57, 105, 122, 251, 213, 189, 176, 7, 253, 152, 174, 228, 51, 53, 94, 211, 22, 149, 222, 10, 2, 184, 201, 101, 75, 157, 79, 215, 107, 129, 172, 17, 243, 234, 134, 194, 249, 90, 124, 131, 219, 112, 43, 169, 236, 225, 80, 179, 45, 98, 208, 167, 60, 40, 207, 155, 15, 204, 36, 159, 164, 161, 186, 103, 198, 142, 239, 240, 146, 28, 12, 209, 69, 156, 39, 85, 56, 160, 74, 141, 91, 23, 114, 14, 238, 128, 143, 247, 16, 32, 70, 216, 250, 178, 44, 202, 197, 42, 88, 190, 223, 61, 193, 9, 100, 224, 231, 254, 59, 183, 121, 92, 99, 37, 170, 50, 171, 195, 102, 148, 205, 220, 136, 95, 151, 46, 133, 154, 241, 49, 192, 153, 180, 13, 87, 81, 191, 244, 0, 97, 89, 227, 200, 187, 221, 73, 24, 96, 150, 230, 106, 67, 218, 252, 11, 130, 217, 20, 126, 166, 4, 104, 140, 145, 226, 206, 248, 110, 118, 245, 233, 132, 33, 52, 158, 27, 31, 64, 125, 203, 93, 63, 138, 144, 214, 162, 66, 117, 229, 196, 123, 188, 21, 62, 135, 84, 165, 3, 235, 139, 185, 115, 29, 8, 212, 71, 210, 113, 127, 199, 41, 182, 54, 242, 1, 108, 26, 173, 168, 255, 35, 163, 175, 78, 58, 18, 147, 38, 82, 177, 76, 237, 6, 119]\\
Fitness: 44922898\\
Problem size: 256\\
Population size: 100\\
Number of generations: 400\\
Individual mutation probability: 0.3\\
Gene mutation probability: 0.002\\
Tournament size: 4\\
Difference from optimal solution: 0.36551961699842717\%\\
Time: 177372\\
Name:data/tai256c.dat\\
\end{tabular}
\caption{Resultados finales}
\end{table}

