\chapter{Conclusiones}


La realización de esta práctica me ha servido como introducción al basto mundo de los algoritmos genéticos. Como nos comentó el profesor, necesitaríamos mucho mas tiempo que un simple cuatrimestre para profundizar mas, pues es un  campo muy amplio. Aun así ha sido muy gratificante poner a prueba los conocimientos adquiridos en la clase de teoría y viendo como los algoritmos van aprendiendo a resolver el problema de forma automática.

\bigskip
Las técnicas que hemos aplicado al problema de la asignación cuadrática nos permiten obtener soluciones razonablemente buenas, pero no óptimas, en un periodo de tiempo de ejecución razonablemente bajo. Estás técnicas son útiles cuando el espacio de búsqueda de nuestro problema es demasiado grande para aplicar técnicas que obtienen soluciones óptimas ya que llevaría demasiado tiempo explorar todo el espacio de búsqueda y dar dicha solución.