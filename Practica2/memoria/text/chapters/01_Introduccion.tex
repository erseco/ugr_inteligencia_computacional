\chapter{Introducción}

El objetivo de esta práctica es comprender el funcionamiento de los algoritmos evolutivos, centrándonos concretamente en los algoritmos genéticos. 

\bigskip
Para ello a nos vamos a enfrentar al problema de la asignación cuadrática o \textbf{QAP} \footnote{Quadratic Assignment Problem} que es un problema fundamental de optimización combinatoria con numerosas aplicaciones. 

\bigskip
El problema se puede describir de la siguiente forma:

\bigskip
Supongamos que queremos decidir dónde construir \textbf{n} instalaciones (p.ej. fábricas) y tenemos \textbf{n} posibles localizaciones en las que podemos construir dichas instalaciones. Conocemos las distancias que hay entre cada par de instalaciones y también el flujo de materiales que ha de existir entre ellas. El problema consiste en decidir dónde ubicar cada fábrica para minimizar el coste de transporte de materiales.

\bigskip
La función de coste seria:

\[ \sum_{i,j} w(i,j) \times d(p(i),p(j)) \]

Siendo:

\begin{itemize}
	\item $ w(i,j) $ el peso asociado al flujo de materiales transportados desde la instalación \textit{i} a la instalación \textit{j}
	\item $ d(i,j) $ la distancia de la localización \textit{i} a la \textit{j}
	\item $ p(i) $ la instalación \textit{i} en una posible solución del problema
\end{itemize}

\bigskip
Los casos de prueba han sido obtenidos de la biblioteca QAPLIB\footnote{\url{http://www.seas.upenn.edu/qaplib/}}. Los ficheros tienen el  formato:
\\ \\
\texttt{n}
\\
\texttt{WWW}\\
\texttt{WWW}\\
\texttt{WWW}
\\
\texttt{DDD}\\
\texttt{DDD}\\
\texttt{DDD}
\\ \\
Donde \textit{n} es el tamaño del problema, \textit{W} es la matriz de flujos de material y \textit{D} es la matriz de distancias .

\bigskip
El objetivo de la práctica es intentar obtener el mejor resultado posible sobre el conjunto de datos de prueba \texttt{tai256c}, como puede deducirse, este problema tiene un espacio de búsqueda de 256!, lo que es igual a $8{,}5781777534284265411908227168123262515778152027948561*10^{506}$, un número lo suficientemente grande como para descartar la posibilidad de resolverlo por fuerza bruta en un tiempo razonable.

\bigskip
Una solución basada en algoritmos genéticos sería una buena aproximación a  esta resolución. Actualmente la mejor solución obtenida con un algoritmo evolutivo para este problema es la de una permutación con un coste de \texttt{44.759.294}.
